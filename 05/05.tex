
\documentclass[13pt,a4paper]{scrartcl}
\usepackage[utf8]{inputenc}
\usepackage[english,russian]{babel}
\usepackage{indentfirst}
\usepackage{graphicx}
\usepackage{amsmath}
\usepackage{amssymb}
\usepackage{listings}
\usepackage{array}

% Некоторые множества

\def\Q{\mathbb{Q}}
\def\Z{\mathbb{Z}}
\def\N{\mathbb{N}}
\def\R{\mathbb{R}}
\def\C{\mathbb{C}}

% Бинарные операции над множествами

% xor
\def\xor{\oplus}
% объединение
\def\u{\cup}
% объединение
\def\i{\cap}


% Комбинаторика

% Биномиальный коэффициент : (из n  по k)
\def\set{\binom}
% ((из n по k))
\def\mset#1#2{\ensuremath{\left(\kern-.3em\left(\genfrac{}{}{0pt}{}{#1}{#2}\right)\kern-.3em\right)}}


% Опеации над несколькими множествами

% Сумма
\def\suml{\sum\limits}

% Сумма
\def\intl{\int\limits}

% Перемножение (знак П)
\def\prodl{\prod\limits}

% Объединение
\def\U{\bigcup}
\def\Ul#1#2#3{\U\limits_{{#1}={#2}}^{#3}}
\def\Ui#1#2{\Ul{#1}{#2}{\inf}}
\def\Uin#1#2{\U\limits_{{#1} \in {#2}}}

% Пересечение
\def\I{\bigcap}
\def\Il#1#2#3{\I\limits_{{#1}={#2}}^{#3}}
\def\Ii#1#2{\Il{#1}{#2}{\inf}}
\def\Iin#1#2{\I\limits_{{#1} \in {#2}}}


% Разделители

\def\ms{\medskip}
\def\bs{\bigskip}


% Греческий алфавит

\def\a{\alpha}
\def\b{\beta}
\def\g{\gamma}
\def\l{\lambda}
\def\e{\varepsilon}
\def\eps{\varepsilon}
\def\d{\delta}
\def\m{\mu}
\def\p{\phi}

\def\L{\Lambda}
\def\D{\Delta}
\def\M{\Mu}
\def\P{\Phi}


% Кванторы

\def\A{\forall}
\def\E{\exists\;}


% Что-то еще

\def\inf{\t{+}\infty}    % +inf
\def\O{\mathcal{O}}      %
\def\t{\text}
\def\bs{\textbackslash{}}


\begin{document}

\def\s{\sigma}
\def\a{\alpha}
\def\X{\overline{X}}
\def\Y{\overline{Y}}

\section*{\text{ Домашнее }\allowbreak \text{задание }\allowbreak \text{по }\allowbreak \text{статистике }\allowbreak 13.10.17}

\subsection*{ 1.}

\(\text{В }\allowbreak \text{предположении, }\allowbreak \text{что }\allowbreak \text{оценки }\allowbreak \text{за }\allowbreak \text{тест }\allowbreak \text{распределены }\allowbreak \text{нормально: }\allowbreak \\\)
\(\dfrac{n S^2 }{\s^2 } \sim  \chi^2(n - 1)\)

\(\text{В }\allowbreak \text{предположении }\allowbreak \text{истинности }\allowbreak \text{нулевой }\allowbreak \text{гипотезы: }\allowbreak \\\)
\(T = \dfrac{n S^2 }{76 } \sim  \chi^2(n - 1)\)

\(\text{Доверительный }\allowbreak \text{интервал }\allowbreak \text{для }\allowbreak \text{статистики }\allowbreak T\text{ при }\allowbreak \alpha = 0.05: (5.63; 26.12)\)

\(T  = \dfrac{15\cdot  110 }{76 } = 21.71053\)

\(\text{Таким }\allowbreak \text{образом, }\allowbreak \text{гипотеза }\allowbreak \text{не }\allowbreak \text{отвергается}\allowbreak \)

\subsection*{ 2.}

\(\X = 47000, n_X = 20, DX = 40000\)

\(\Y = 52000, n_Y = 30, DY = 90000\)

\(T = (\X - \Y)\cdot  (\frac{DX }{n_X} + \frac{DY }{n_Y})^{-1/2} \sim  N(0, 1)\)

\(\hat T = (47000 - 52000)\cdot  (40000 / 20 + 90000 / 30)^{-1/2} = -70.71068\)

\(\text{при }\allowbreak \a = 0.01,\text{ допустимая }\allowbreak \text{ошибка }\allowbreak error = 2.575829\)

\(|\hat T| > error \Rightarrow \text{ гипотеза }\allowbreak \text{о }\allowbreak \text{равенстве }\allowbreak \text{средних }\allowbreak \text{отвергается}\allowbreak \)

\subsection*{ 3.}

\def\delim{C_i / 10002}

\(T = 10002\cdot  \suml_{i = 0}^{9} \dfrac{(\delim - 0.1)^2 }{0.1 } \sim  \chi^2(9)\)

\(\hat T = 9.367726\)

\(\text{В }\allowbreak \text{данном }\allowbreak \text{случае }\allowbreak \text{критическая }\allowbreak \text{область }\allowbreak \text{будет }\allowbreak \text{справа, }\allowbreak \text{соответственно, }\allowbreak \text{легко }\allowbreak \text{вычислить}\allowbreak \)
\(pvalue = 1 - F_T(\hat T) = 0.404\)

\(\text{Таким }\allowbreak \text{образом, }\allowbreak \text{гипотеза }\allowbreak \text{не }\allowbreak \text{отвергается }\allowbreak \text{при }\allowbreak \a < 0.404,\)
\(\text{и }\allowbreak \text{отвергается }\allowbreak \text{при }\allowbreak \a \ge  0.404\)



\end{document}
